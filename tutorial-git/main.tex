\documentclass[a4paper, 12pt]{lnls-note}
%\documentclass[a4paper, 12pt, report]{lnls-note} % if you want to have chapters choose this one

\usepackage{listings}

\definecolor{mygray}{rgb}{0.9,0.9,0.9}

\lstset{
language=bash,
basicstyle=\ttfamily\footnotesize,
backgroundcolor=\color{mygray}, 
showstringspaces=false}


%loads standard preamble configuration
\input{standard_preamble.tex}

%loads standard commands
\input{commands.tex}


\begin{document}

\lnlstitle
{04 de Janeiro de 2018}
{Introdução ao Git}
{Luana Nayara Pires Vilela}
{\LNLS}
{Neste tutorial são apresentados alguns aspectos básicos da utilização do sistema de controle de versões Git e da plataforma GitHub.}
\setcounter{page}{2}
\newpage

\section{Sistema de controle de versões}
Um sistema de controle de versões é um software utilizado para gerenciar diferentes versões de arquivos, mantendo o histórico de suas versões antigas e os registros de quem e quando manipulou estes arquivos. 

Estes sistemas são utilizados principalmente para desenvolvimento de softwares, pois facilitam o compartilhamento e a recuperação de todas as versões dos arquivos, além de possibilitar que diversas pessoas trabalhem simultaneamente no mesmo conjunto de documentos. 

\section{Git e SVN}
O Git e o Subversion (SVN) são dois exemplos de sistemas de controle de versões. Em ambos os sistemas os dados de todas as versões dos arquivos são salvos em banco de dados. O local onde este banco de dados fica armazenado é chamado de repositório. 

A principal diferença entre os dois sistemas é que o SVN é um sistema servidor-cliente centralizado enquanto o Git é um sistema distribuído. Ou seja, no caso do SVN existe um repositório central localizado em um servidor a partir do qual os clientes podem copiar a versão desejada dos arquivos (para isso os computadores devem estar conectados a mesma rede). No caso do Git cada cópia local contém todas as versões dos arquivos daquele projeto, ou seja, cada cópia é um repositório completo. Assim, a maioria das ações do Git podem ser executadas offline enquanto que no SVN a maioria das ações depende da conexão com o servidor central, o que faz com que este último sistema tenha menor desempenho. Além disso, o Git é menos susceptível a falhas, pois se houver algum problema com o servidor, todos os clientes têm uma cópia completa do repositório que pode ser disponibilizada novamente.

\section{GitHub}
Outra vantagem de usar o Git é que existe uma plataforma de hospedagem de repositórios chamada \href{https://github.com/}{GitHub} que possibilita que qualquer pessoa conectada a internet possa ter acesso aos repositórios. Assim, os computadores não precisam estar conectados a uma mesma rede para que os arquivos sejam compartilhados. No entanto, como os repositórios no GitHub são públicos e qualquer pessoa pode ter acesso as informações contidas neles, esta ferramenta só deve ser utilizada para códigos abertos e informações não sigilosas. 

O grupo de imãs possui uma página no GitHub chamada \href{https://github.com/lnls-ima}{lnls-ima} que atualmente contém repositórios para cada um dos ímãs desenvolvidos para o Sirius, assim como alguns repositórios com código-fonte de softwares de medições magnéticas e análises de medidas. 

O objetivo deste tutorial é facilitar o uso do Git e do GitHub para aqueles que não estão familiarizados com essas ferramentas para que todos os softwares de medidas e análises implementados pelo grupo de imãs sejam disponibilizados na página do grupo no GitHub. 

Para conseguir criar e modificar repositórios do grupo de imãs é preciso criar uma conta no \href{https://github.com/}{GitHub}, ser incluído na organização lnls-ima (isso deve ser feito por algum administrador da organização) e instalar o Git no seu computador.

\section{Instalação do Git}

\subsection*{Instalação no Windows}

Recomenda-se a instalação da interface gráfica do GitHub para Windows, encontrada \href{https://desktop.github.com/}{aqui}. 

Se você pretende usar apenas a interface gráfica não é necessário instalar o Git. No entanto, se você quiser usar a linha de comando precisará instalar a versão do Git para o Windows, que pode ser encontrada \href{http://gitforwindows.org/}{aqui}.

\subsection*{Instalação no Linux}

Se você usa uma distribuição baseada em Debian, pode instalar o Git usando o apt-get:

\begin{lstlisting}
  $ sudo apt-get install git-all
\end{lstlisting}

Para outras distribuições veja as instruções \href{https://git-scm.com/download/linux}{aqui}.

\section{Configuração do Git}

Após a instalação do Git é preciso configurar o seu nome de usuário e endereço de e-mail. Esta configuração pode ser feita utilizando a interface gráfica (GitHub Desktop) ou utilizando a linha de comando.

\subsection*{Configuração com a interface gráfica}

Ao abrir pela primeira vez a interface do GitHub Desktop clique em "Sign into GitHub.com". Entre com o seu usuário e senha do GitHub. Em seguida a tela de configuração aparecerá. Preencha o seu nome e e-mail e clique no botão "Continue".

\subsection*{Configuração com a linha de comando}

Para configurar o nome de usuário e e-mail utilize os seguintes comandos:

\begin{lstlisting}
  $ git config --global user.name "John Doe"
  $ git config --global user.email johndoe@example.com
\end{lstlisting}

\section{Usando o Git}

Nesta seção são apresentados alguns dos comandos básicos necessários para a utilização do Git. Para entender um pouco melhor como utilizar esta ferramenta pode ser feito o curso \href{https://www.codecademy.com/learn/learn-git}{Learn Git} disponível gratuitamente na plataforma Codecademy. Para aprofundar seu conhecimento sobre o assunto recomenda-se o livro \href{https://git-scm.com/book/en/v2}{Pro Git}, que contêm uma descrição detalhada dos comandos e recursos disponíveis no Git. 

\subsection*{Interface gráfica}

A interface do GitHub Desktop, mostrada na figura a seguir, contem as funcionalidades básicas do Git e possibilita o uso deste sistema sem a linha de comando. Na \hyperref[example]{seção seguinte} é apresentado um exemplo de utilização desta interface.

\begin{figure}[H]
\begin{center}
\includegraphics[width=0.9\textwidth]{ghd.png}
\end{center}
\end{figure}

\subsection*{Linha de comando}

A seguir é apresentada uma breve descrição de alguns dos principais comandos do Git. Ao utilizar os comandos abaixo, o texto entre <> deve ser substituído pelos nomes apropriados.

\begin{description}[style=nextline]

\item[git init] 
Inicializa um novo repositório no local onde o comando foi executado.  

\item[git clone git@github.com:lnls-ima/<repository>.git] 
Cria uma cópia do repositório do GitHub no local onde o comando foi executado. 

\item[git pull] Traz as mudanças feitas no repositório remoto para o repositório local.

\item[git status] Mostra quais arquivos foram modificados desde a última vez que o banco de dados local foi atualizado.

\item[git diff] Mostra as mudanças feitas nos arquivos.

\item[git add <filename>] Seleciona o arquivo <filename> para que as mudanças feitas neste arquivo sejam incluídas no banco de dados local na próxima atualização.

\item[git commit -m “<message>”] Atualiza o banco de dados local. O termo <message> deve ser substituído por uma mensagem explicando quais as mudanças realizadas desde a última atualização.

\item[git push] Envia as mudanças feitas no repositório local para o repositório remoto.

\end{description}

Um resumo de todos os comandos do Git pode ser encontrado  \href{http://ndpsoftware.com/git-cheatsheet.html}{aqui}. 

\section{Exemplo}
\label{example}

Nesta seção é mostrado como modificar um arquivo de um repositório do GitHub utilizando a interface gráfica ou a linha de comando. Neste exemplo o arquivo \textit{README.md} do repositório chamado \textit{bo-correctors} será modificado.

\subsection*{Interface gráfica}

Inicialmente é preciso clonar o repositório \textit{bo-correctors} e assim ter cópia dos arquivos no seu computador. Isto pode ser feito clicando no botão "Clone a repository" da página inicial do GitHub Desktop ou a partir do menu "File" $\rightarrow$ "Clone repository". A seguinte tela será mostrada:

\begin{figure}[H]
\begin{center}
\includegraphics[width=0.9\textwidth]{ex1.png}
\end{center}
\end{figure}

Selecione o repositório desejado e o local onde a cópia deste repositório será armazenada ("Local Path"). Em seguida clique em "Clone".

Inicialmente os arquivos do repositório local serão iguais aos arquivos do repositório remoto do GitHub. Para abrir a pasta que contém os arquivos clique em "open this repository", conforme indicado na figura abaixo:

\begin{figure}[H]
\begin{center}
\includegraphics[width=0.9\textwidth]{ex2.png}
\end{center}
\end{figure}

Faça as mudanças desejadas. Neste caso, uma das linhas do arquivo \textit{README.md} será apagada, conforme mostrado na figura a seguir:

\begin{figure}[H]
\begin{center}
\includegraphics[width=\textwidth]{ex3.png}
\end{center}
\end{figure}

A interface do GitHub mostra a lista dos arquivos modificados e as mudanças feitas em cada arquivo. Para salvar a nova versão dos arquivos no banco de dados local adicione uma mensagem explicando as alterações no campo "Summary" e no campo "Description" (opcional) e em seguida clique em "Commit to master":

\begin{figure}[H]
\begin{center}
\includegraphics[width=0.9\textwidth]{ex4.png}
\end{center}
\end{figure}

Agora que as mudanças estão salvas no repositório local elas podem ser enviadas para o repositório remoto. Para isso clique no botão "Push origin" mostrado abaixo: 

\begin{figure}[H]
\begin{center}
\includegraphics[width=0.9\textwidth]{ex5.png}
\end{center}
\end{figure}

Pronto! Agora todas as pessoas que utilizam este repositório poderão atualiza-lo clicando no botão "Fetch origin" e em seguida no botão "Pull origin" e o arquivo será corrigido em todas as cópias. O commit realizado e a mensagem descritiva são mostrados no repositório do GitHub:

\begin{figure}[H]
\begin{center}
\includegraphics[width=\textwidth]{ex6.png}
\end{center}
\end{figure}

\subsection*{Linha de comando}

Para modificar o arquivo primeiramente é necessário obter uma cópia do repositório usando o comando \textbf{clone}:

\begin{lstlisting}
  $ git clone git@github.com:lnls-ima/bo-correctors.git
Cloning into 'bo-correctors'...
remote: Counting objects: 5470, done.
remote: Compressing objects: 100% (3143/3143), done.
remote: Total 5470 (delta 86), reused 3186 (delta 41), pack-reused 2200
Receiving objects: 100% (5470/5470), 65.10 MiB | 6.26 MiB/s, done.
Resolving deltas: 100% (213/213), done.
Checking connectivity... done.  
\end{lstlisting}

Em seguida são feitas as mudanças desejadas nos arquivos. Para verificar quais arquivos foram modificados utiliza-se o comando \textbf{status}:

\begin{lstlisting}
  $ git status
On branch master
Your branch is up-to-date with 'origin/master'.
Changes not staged for commit:
  (use "git add <file>..." to update what will be committed)
  (use "git checkout -- <file>..." to discard changes in working 
directory)

	modified:   README.md

no changes added to commit (use "git add" and/or "git commit -a")  
\end{lstlisting}

Para selecionar o arquivo para que as mudanças realizadas nele sejam incluídas no banco de dados local utiliza-se o comando \textbf{add}:

\begin{lstlisting}
  $ git add README.md
\end{lstlisting}

Em seguida utiliza-se o comando \textbf{commit} para atualizar o banco de dados local:

\begin{lstlisting}
  $ git commit -m "Remove line from README"
[master 6b31236] Remove line from README
 1 file changed, 1 deletion(-)  
\end{lstlisting}

Por fim, as mudanças feitas no repositório local são enviadas para o repositório remoto utilizando o comando \textbf{push}:

\begin{lstlisting}
  $ git push
Counting objects: 3, done.
Delta compression using up to 8 threads.
Compressing objects: 100% (2/2), done.
Writing objects: 100% (3/3), 302 bytes | 0 bytes/s, done.
Total 3 (delta 2), reused 1 (delta 1)
remote: Resolving deltas: 100% (2/2), completed with 2 local objects.
To git@github.com:lnls-ima/bo-correctors.git
   54c32c7..6b31236  master -> master  
\end{lstlisting}

\section{Links uteis}

\begin{description}
\item[\href{http://gitforwindows.org/}{Git for Windows}] Versão do Git para Windows.
\item[\href{https://desktop.github.com/}{GitHub Desktop}] Interface gráfica do GitHub para Windows.
\item[\href{http://ndpsoftware.com/git-cheatsheet.html}{Git CheatSheet}] Planilha com o resumo dos comandos do Git.
\item[\href{https://git-scm.com/book/en/v2}{Pro Git}] Livro bem detalhado sobre o Git.
\item[\href{https://www.codecademy.com/learn/learn-git}{Learn Git}] Curso disponível no Codecademy.
\end{description}

% % Our bibtex file :)
% \bibliographystyle{unsrt}
% \bibliography{references}


\end{document}









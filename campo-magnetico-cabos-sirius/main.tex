\documentclass[a4paper, 12pt]{lnls-note}
%\documentclass[a4paper, 12pt, report]{lnls-note} % if you want to have chapters choose this one

%loads standard preamble configuration
\input{standard_preamble.tex}

\usepackage{booktabs}
\usepackage{multirow}
\usepackage{floatrow}
\floatsetup[table]{capposition=top}

%loads standard commands
\input{new_commands.tex}
\setlength{\tabcolsep}{12pt}

\begin{document}

\lnlstitle{28 de Fevereiro de 2018}{Simulações do campo magnético gerado pelos cabos das fontes de corrente do Sirius}
{Luana Vilela}{\LNLS}
{Neste relatório são apresentados os resultados das simulações magnéticas para o campo gerado pelos cabos das fontes de corrente dos imãs na região em torno do feixe de elétrons tanto do Booster como do anel de armazenamento do Sirius.}

\setcounter{page}{2}
\newpage

\section*{Simulações magnéticas}
As simulações magnéticas foram feitas com o software MagNet. O modelo utilizado nas simulações é mostrado na figura \ref{fig:modelo_2d}. O aço CR10 (Cold rolled 1010 steel) foi usado para os materiais das calhas e berços. Devido a distância entre os berços não há influência entre os campos gerados pelos cabos do anel e do Booster, por isso foram feitas simulações separadas para cada acelerador. Os valores de corrente utilizados para cada família de ímã são apresentados nas tabelas~\ref{table:bo_correntes}~e~\ref{table:si_correntes}.   

\begin{figure}[H]
\begin{center}
\includegraphics[width=0.9\textwidth]{modelo2D.png}
\caption{Modelo 2D dos componentes utilizados nas simulações.}
\label{fig:modelo_2d}
\end{center}
\end{figure}

\begin{table}[H]
\centering
\caption{Corrente nominal de extração dos ímãs do Booster.}
\begin{tabular}{cc}
\toprule
\textbf{Família do ímã} & \textbf{Corrente [A]} \\ 
\midrule
BD &  1034.0\\
BQF & 114.0\\
BQD & 30.4\\
BS & 142.0\\
\bottomrule
\end{tabular}
\label{table:bo_correntes}
\end{table}

\begin{table}[H]
\centering
\caption{Corrente nominal dos ímãs do anel de armazenamento.}
\begin{tabular}{cc}
\toprule
\textbf{Família do ímã} & \textbf{Corrente [A]} \\ 
\midrule
B1/B2 & 380.00 \\
QFA  & 122.45  \\
QFB  & 139.81  \\
QFP  & 139.81  \\
QDA  & -64.12  \\
QDB1 & -79.34  \\
QDB2 & -135.09 \\
QDP1 & -79.34  \\
QDP2 & -135.09 \\
Q1   & 96.49   \\
Q2   & 148.81  \\
Q3   & 110.88  \\
Q4   & 134.81  \\ 
SDA0 & -53.38  \\
SDB0 & -42.88  \\
SDP0 & -42.88  \\
SFA0 & 34.71   \\
SFB0 & 48.69   \\
SFP0 & 48.69   \\
SDA1 & -107.62 \\
SDA2 & -58.67  \\
SDA3 & -92.41  \\
SFA1 & 126.67  \\
SFA2 & 99.56   \\
SDB1 & -93.52  \\
SDB2 & -80.71  \\
SDB3 & -114.77 \\
SFB1 & 150.47  \\
SFB2 & 130.62  \\
SDP1 & -93.97  \\
SDP2 & -80.74  \\
SDP3 & -115.00 \\
SFP1 & 151.38  \\
SFP2 & 131.08  \\
\bottomrule
\end{tabular}
\label{table:si_correntes}
\end{table}

\newpage

\section*{Booster}
As figuras~\ref{fig:bo}~e~\ref{fig:bo_field} mostram a distribuição dos cabos de alimentação das fontes do Booster e os resultados das simulações magnéticas. A média do campo magnético campo vertical no eixo é de \SI{0.28}{G}. Considerando esse valor constante ao longo de todo o comprimento do Booster obtemos um erro de deflexão de aproximadamente \SI{1.4}{\milli\radian} devido ao campo gerado pelos cabos de alimentação.

\begin{figure}[H]
\begin{center}
\includegraphics[width=\textwidth]{bo.png}
\caption{Distribuição de correntes nos cabos das fontes de alimentação do Booster.}
\label{fig:bo}
\end{center}
\end{figure}

\begin{figure}[H]
\begin{center}
\includegraphics[width=0.7\textwidth]{bo_field.png}
\caption{Campo magnético gerado pelos cabos das fontes dos ímãs do Booster no plano da órbita dos elétrons.}
\label{fig:bo_field}
\end{center}
\end{figure}

\newpage

\section*{Anel de armazenamento}

\subsection*{Pior Caso}
 Inicialmente foi considerado o caso em que todas as corrente tem o mesmo sentido para testar a influência dos cabos no campo magnético em torno da posição de referência do feixe de elétrons. A soma total da corrente neste caso é de \si{3284}{A}. Os resultados são apresentados na figura~\ref{fig:si_pior_caso}.

\begin{figure}[H]
\begin{center}
\includegraphics[width=\textwidth]{si_pior_caso.png}
\caption{Campo magnético gerado pelos cabos das fontes dos ímãs do anel de armazenamento no caso em que todas as correntes tem o mesmo sentido (pior caso).}
\label{fig:si_pior_caso}
\end{center}
\end{figure}

\subsection*{Caso Otimizado}
O otimização do campo magnético na região próxima ao feixe foi feita escolhendo a distribuição de correntes que minimizava a soma total das correntes dos cabos. O valor mínimo obtido foi de \SI{1.2}{A}. Os resultados para este caso otimizado são mostrados na figura~\ref{fig:si_otimizado} e a distribuição de correntes é apresentada na figura~\ref{fig:si_otimizado_cabos}.

\begin{figure}[H]
\begin{center}
\includegraphics[width=\textwidth]{si_otimizado.png}
\caption{Campo magnético gerado pelos cabos das fontes dos ímãs do anel de armazenamento no caso otimizado.}
\label{fig:si_otimizado}
\end{center}
\end{figure}

\begin{figure}[H]
\begin{center}
\includegraphics[width=0.9\textwidth]{si_otimizado_cabos.png}
\caption{Distribuição otimizada de correntes nos cabos das fontes do anel de armazenamento.}
\label{fig:si_otimizado_cabos}
\end{center}
\end{figure}

A tabela~\ref{tab:si} mostra os valores médios do campo magnético na região [-12, 12] mm em torno do feixe para ambos os casos citados. A estimativa para o erro de deflexão total no anel de armazenamento é de \SI{0.7}{\milli\radian} para a distribuição otimizada de correntes.

\begin{table}[hbt]
\centering
\caption{Campo magnético gerado pelos cabos da fonte de corrente do anel de armazenamento.}
\begin{tabular}{lcccc}
\toprule
 & $B_{x_{avg}} [G]$ & $B_{y_{avg}} [G]$ \\ 
\midrule
Pior caso & -2.2  & -7.5 \\
Caso otimizado & -0.02 & -0.14\\
\bottomrule
\end{tabular}
\label{tab:si}
\end{table}

\qquad

\textbf{Observação:} Nas simulações do anel de armazenamento o suporte do berço foi considerado um bloco maciço de aço (conforme mostrado na figura~\ref{fig:modelo_2d}). No entanto a parte interna do suporte será composta principalmente de concreto, assim, é esperado que o campo magnético próximo ao feixe de elétrons seja menor que os valores obtidos com este modelo. 

\end{document}









\documentclass[a4paper, 12pt]{lnls-note}
%\documentclass[a4paper, 12pt, report]{lnls-note} % if you want to have chapters choose this one

\renewcommand\footnoterule{}
%loads standard preamble configuration
\input{standard_preamble.tex}

\usepackage{booktabs}
\usepackage{multirow}
\usepackage{floatrow}
\usepackage{subcaption}
\usepackage[bottom]{footmisc}
\floatsetup[table]{capposition=top}

%loads standard commands
\input{new_commands.tex}

\begin{document}

\lnlstitle{14 de fevereiro de 2019}{Calibração dos Sensores Hall}
{Luana Vilela, Reinaldo Basilio, James Citadini e Leandro Silveira}{\LNLS}
{Este relatório apresenta os resultados das medidas de calibração dos sensores Hall fabricados pela SENIS, tipo H3A-0xI02L-G02T0K4L com número de série 315-14.}
\setcounter{page}{2}
\newpage

\section{Introdução}
    A caracterização de ímãs com sensores Hall é feita a partir da medida da voltagem induzida no sensor na presença de um campo magnético. O valor da voltagem induzida depende linearmente do campo magnético para valores não muito altos de campo: a sensibilidade dos sensores especificada pelo fabricante é de \SI{5}{\volt/\tesla} para campos de até \SI{2}{\tesla}. Para campo maiores, a dependência com a voltagem deixa de ser linear. Por isso é necessário utilizar uma curva de calibração para determinar o valor de campo correspondente a cada valor de tensão. Para construir a curva de calibração do sensor é necessário conhecer os valores do campo magnético antecipadamente ou medi-los utilizando outros métodos.
    
\section{Setup das medidas de calibração}  
    As medidas de calibração foram feitas com ímã supercondutor~(BSC) de \SI{6}{\tesla} da Cryomagnetics instalado na linha de luz PGM (Figura~\ref{fig:bsc}).
    
    \begin{figure}[H]
        \begin{center}
        \includegraphics[width=0.65\textwidth]{bsc.jpg}
        \caption{Ímã supercondutor da linha de luz PGM, fabricado pela Cryomagnetics.}
        \label{fig:bsc}
        \end{center}
    \end{figure}
    
    Para as medidas da voltagem foi utilizado um multímetro Agilent 3458A. A configuração utilizada corresponde a configuração default do equipamento, exceto pelo tempo de integração, para o qual foi utilizado o valor de \SI{100}{\milli\second}. 
    
    Durante as medidas, foram monitoradas as temperaturas do sensor, do transdutor e a temperatura dentro do ímã usando um PT100 e um multicanal Agilent 34970A.
    
    A ideia inicial era comparar o valor do campo magnético retornado pela fonte do ímã supercondutor com medidas de NMR (Figura~\ref{fig:rack}). No entanto, não foi possível obter medidas com o NMR, devido à baixa homogeneidade (0.5 \%/cm) e à alta flutuação ($\pm$~5~G) do campo magnético gerado pelo supercondutor. Por esse motivo a curva de calibração dos sensores Hall foi construída utilizando os valores de setpoint do campo magnético da fonte do ímã supercondutor.
    
    \begin{figure}[H]
        \begin{center}
        \includegraphics[width=0.7\textwidth]{rack.jpg}
        \caption{Rack com os equipamentos utilizados durante as medidas de calibração.}
        \label{fig:rack}
        \end{center}
    \end{figure}
    
    Foram medidas as curvas de calibração dos sensores X, Y e Z do transdutor SENIS tipo H3A-0xI02L-G02T0K4L com número de série 315-14. Cada sensor foi medido separadamente para facilitar o ajuste da posição do sensor dentro do ímã. Os sensores foram colocados dentro da bobina supercondutora utilizando a haste mostrada na Figura~\ref{fig:haste}.
    
    \begin{figure}[H]
        \center
        \subfigure[][]{\includegraphics[width=\textwidth]{haste.png}}
        \qquad
        \subfigure[][]{\includegraphics[width=0.7\textwidth]{haste_bsc.png}}
        \caption{(a)~Haste utilizada para posicionar os sensores Hall e o sensor de temperatura dentro do ímã supercondutor. (b)~Detalhe do posicionamento da haste no ímã supercondutor.}
        \label{fig:haste}
    \end{figure}
    
    A componente linear da curva de calibração dos sensores X e Y também foi medida utilizando o dipolo de calibração do grupo Ímãs, que atingi campo magnéticos de até \SI{2}{\tesla} (vide Figura~\ref{fig:dipolo_calibracao}). Neste dipolo a homogeneidade do campo é boa o suficiente para conseguir fazer a checagem com o NMR.

    \begin{figure}[H]
        \begin{center}
        \includegraphics[width=0.7\textwidth]{dipolo_calibracao.png}
        \caption{Dipolo de calibração de \SI{2}{\tesla}.}
        \label{fig:dipolo_calibracao}
        \end{center}
    \end{figure}

\section{Resultados das medidas}
    Nas seções seguintes são apresentados os resultados obtidos nas medidas de calibração. Os dados coletados estão na pasta: 

    \url{\\\\centaurus\\Repositorio\\LNLS\\Grupos\\IMA\\Sirius\\Projetos\\SIRIUS-Magnets\\Magnets-Measuring_System\\Calibration_Measurements\\2019-01-28}
    
    \subsection{Estabilidade do campo magnético}
        Durante as medidas de calibração foi observada uma flutuação no campo magnético gerada pela fonte de alimentação do ímã supercondutor. A Figura~\ref{fig:estab} mostra uma medida realizada com o sensor Hall tipo Y durante \SI{10}{\min} com o setpoint da fonte fixo em \SI{-3.5}{\tesla}. O desvio padrão dos dados coletados é de \SI{1}{mV}, equivalente à aproximadamente \SI{2}{G}. A variação de pico a pico é de aproximadamente \SI{10}{G}. Devido a esta flutuação do campo as medidas da curva de calibração foram realizadas utilizando a média dos valores de voltagem coletados durante aproximadamente \SI{1}{\min} à uma taxa de \SI{2}{\hertz}.
        \begin{figure}[H]
            \begin{center}
            \includegraphics[width=0.9\textwidth]{estabilidade.png}
            \caption{Voltagem medida com o sensor Hall tipo Y mantendo o setpoint da fonte de alimentação do supercondutor fixo.}
            \label{fig:estab}
            \end{center}
        \end{figure}
    
    \subsection{Sensor Y}
        A Figura~\ref{fig:sensory} mostra o posicionamento do sensor Y na haste de medição. A curva de calibração para este sensor é apresentada na Figura~\ref{fig:caly}. Durante a medida a temperatura dentro do ímã variou de \SI{20.8}{\celsius} à \SI{22.8}{\celsius}.
        \begin{figure}[H]
            \begin{center}
            \includegraphics[width=0.5\textwidth]{sensory.png}
            \caption{Posicionamento do sensor Y na haste.}
            \label{fig:sensory}
            \end{center}
        \end{figure}
        
        \begin{figure}[H]
            \begin{center}
            \includegraphics[width=0.9\textwidth]{caly.png}
            \caption{Curva de calibração do sensor Y tipo H3A-0xI02L-G02T0K4L SN 315-14 medida no ímã supercondutor.}
            \label{fig:caly}
            \end{center}
        \end{figure}
        
        A curva de calibração dos sensores deveria ser simétrica para campos magnéticos positivos e negativos. No entanto, foram observadas diferenças da ordem de \SI{80}{G} entre os lados positivos e negativos das curvas medidas. Para gerar curvas de calibração simétricas foi feita uma média entre as medidas para valores positivos e negativos de campo. As funções de calibração para os três sensores foram obtidas com um ajuste polinomial desta curva média. O melhor ajuste foi obtido com polinômios impares até ordem 11~(vide equação~\ref{eq:fit}). 
        \begin{align}
            B = K_1 V + K_3 V^3 + K_5 V^5 + ... + K_{11} V^{11},
            \label{eq:fit}
        \end{align}
        onde $B$ é o campo magnético e $V$ é a voltagem do sensor Hall.
        
        Para checar o valor do coeficiente linear desta curva ($K_1$) foram realizadas medidas no dipolo de calibração de \SI{2}{\tesla} utilizando o NMR. Os resultados são mostrados na Figura~\ref{fig:caly_dipole}. A amplitude do coeficiente linear para estas medidas foi de \SI{0.19996}{\tesla/\volt}, bem próximo do valor \SI{0.2}{\tesla/\volt} especificado pela SENIS.
         \begin{figure}[H]
            \begin{center}
            \includegraphics[width=0.9\textwidth]{caly_dipole.png}
            \caption{Parte linear da curva de calibração do sensor Y medida com o NMR e com o dipolo de calibração de \SI{2}{\tesla}.}
            \label{fig:caly_dipole}
            \end{center}
        \end{figure}       
        
        A curva final de calibração foi obtida rotacionando a curva média obtida com o ímã supercondutor até que o coeficiente linear fosse igual ao medido com o NMR. Esta análise foi feita utilizando apenas os dados no intervalo~[-1.6, 1.6]~\si{\tesla}. O ângulo de rotação utilizado para correção da curva do sensor~Y foi de \SI{-0.873}{\milli\radian}. Os coeficientes da curva de calibração final para o sensor~Y são apresentados na tabela~\ref{tab:sensory}. A diferença entre a curva de calibração final e os dados brutos é mostrada na Figura~\ref{fig:caly_error}.
        \begin{table}[hbt]
            \centering
            \caption{Ajuste polinomial da curva de calibração do sensor Y.}
            \begin{tabular}{cc}
            \toprule
            \textbf{Coeficiente} & \textbf{Valor} \\ 
            \midrule
            $K_1$  &  -1.9960e-01   \\ 
            $K_3$  &  -3.6948e-05   \\
            $K_5$  &   1.0317e-06   \\
            $K_7$  &  -9.6282e-09  \\
            $K_9$  &   2.8003e-11   \\
            $K_{11}$  & -3.3634e-14  \\
            \bottomrule
            \end{tabular}
            \label{tab:sensory}
        \end{table}

         \begin{figure}[H]
            \begin{center}
            \includegraphics[width=0.9\textwidth]{caly_error.png}
            \caption{Diferença entre a curva de calibração final do sensor~Y e dados brutos medidos com o ímã supercondutor.}
            \label{fig:caly_error}
            \end{center}
        \end{figure}       
        
    \subsection{Sensor X}
        Os resultados para a curva de calibração do sensor X\footnote{Neste relatório foi adotada a convenção da SENIS na nomeação dos sensores. O sensor aqui chamado de X é aquele com a etiqueta do fabricante para a componente X. No entanto, nas medidas realizadas na bancada Hall este sensor é utilizado para medir a componente Z do campo magnético.} são mostrados na Figura~\ref{fig:calx}. Durante as medidas a temperatura dentro do ímã supercondutor variou de \SI{20.8}{\celsius} à \SI{23.8}{\celsius}.
        \begin{figure}[H]
            \begin{center}
            \includegraphics[width=0.5\textwidth]{sensorx.png}
            \caption{Posicionamento do sensor X na haste utilizada nas medições.}
            \label{fig:sensorx}
            \end{center}
        \end{figure}
        
        \begin{figure}[H]
            \begin{center}
            \includegraphics[width=0.9\textwidth]{calx.png}
            \caption{Curva de calibração do sensor X tipo H3A-0xI02L-G02T0K4L SN 315-14 medida no ímã supercondutor.}
            \label{fig:calx}
            \end{center}
        \end{figure}
    
        O valor do coeficiente linear da curva de calibração do sensor~X também foi verificado com medidas do dipolo de calibração~(vide Figura~\ref{fig:calx_dipole}). A amplitude do coeficiente linear obtida com as medidas do dipolo para este sensor foi de \SI{0.19972}{\tesla/\volt}.
         \begin{figure}[H]
            \begin{center}
            \includegraphics[width=0.9\textwidth]{calx_dipole.png}
            \caption{Parte linear da curva de calibração do sensor X medida com o NMR e com o dipolo de calibração de \SI{2}{\tesla}.}
            \label{fig:calx_dipole}
            \end{center}
        \end{figure}      

        O ângulo de correção aplicado a curva de calibração do sensor~X foi de \SI{0.699}{\milli\radian}.  Os coeficientes da curva de calibração final para o sensor~X são apresentados na tabela~\ref{tab:sensorx} e a diferença entre a curva de calibração final e os dados brutos é mostrada na Figura~\ref{fig:calx_error}.
        \begin{table}[hbt]
            \centering
            \caption{Ajuste polinomial da curva de calibração do sensor~X.}
            \begin{tabular}{cc}
            \toprule
            \textbf{Coeficiente} & \textbf{Valor} \\ 
            \midrule
            $K_1$    &  1.9933e-01   \\ 
            $K_3$    &  4.1553e-05   \\
            $K_5$    & -1.1424e-06    \\
            $K_7$    &  1.0130e-08  \\
            $K_9$    & -2.9328e-11     \\
            $K_{11}$ &  3.4512e-14  \\
            \bottomrule
            \end{tabular}
            \label{tab:sensorx}
        \end{table}

         \begin{figure}[H]
            \begin{center}
            \includegraphics[width=0.9\textwidth]{calx_error.png}
            \caption{Diferença entre a curva de calibração final do sensor~X e dados brutos medidos com o ímã supercondutor.}
            \label{fig:calx_error}
            \end{center}
        \end{figure} 

    \subsection{Sensor Z}
        Os resultados para a curva de calibração do sensor Z são mostrados na Figuras~\ref{fig:calz}. Durante as medidas a temperatura dentro do ímã supercondutor variou de \SI{22.9}{\celsius} à \SI{24.2}{\celsius}.
        \begin{figure}[H]
            \begin{center}
            \includegraphics[width=0.4\textwidth]{sensorz.png}
            \caption{Posicionamento do sensor Z na haste.}
            \label{fig:sensorz}
            \end{center}
        \end{figure}
        
        \begin{figure}[H]
            \begin{center}
            \includegraphics[width=0.9\textwidth]{calz.png}
            \caption{Curva de calibração do sensor Z tipo H3A-0xI02L-G02T0K4L SN 315-14  medida no ímã supercondutor.}
            \label{fig:calz}
            \end{center}
        \end{figure}
        
        O sensor~Z não pode ser medido utilizando o dipolo de calibração pois o gap deste dipolo não era grande o suficiente para acomodar o sensor. Assim, o coeficiente linear da curva de calibração não foi corrigido. Os coeficientes da curva de calibração final para o sensor~Z são apresentados na tabela~\ref{tab:sensorz} e a diferença entre a curva de calibração final e os dados brutos é mostrada na Figura~\ref{fig:calz_error}.
        \begin{table}[hbt]
            \centering
            \caption{Ajuste polinomial da curva de calibração do sensor~Z.}
            \begin{tabular}{cc}
            \toprule
            \textbf{Coeficiente} & \textbf{Valor} \\ 
            \midrule
            $K_1$    &   1.9985e-01  \\ 
            $K_3$    &   3.3025e-05  \\
            $K_5$    &   -9.2560e-07  \\
            $K_7$    &    8.9590e-09  \\
            $K_9$    &   -2.6220e-11  \\
            $K_{11}$ &   3.1928e-14   \\
            \bottomrule
            \end{tabular}
            \label{tab:sensorz}
        \end{table}

         \begin{figure}[H]
            \begin{center}
            \includegraphics[width=0.9\textwidth]{calz_error.png}
            \caption{Diferença entre a curva de calibração final do sensor~Z e dados brutos medidos com o ímã supercondutor.}
            \label{fig:calz_error}
            \end{center}
        \end{figure} 

\section{Resumo dos resultados}

    Os coeficientes do ajuste polinomial das curvas de calibração dos sensores~X,~Y~e~Z são apresentados na tabela~\ref{tab:sensores}. No entanto, as medidas feitas com o ímã supercondutor apresentaram uma discrepância de até \SI{200}{G} em relação as medidas feitas no dipolo de calibração com o NMR e a correção aplicada pode não ser válida para a parte não linear da curva. Portanto, métodos mais precisos precisam ser utilizados para a calibração destes sensores.    
    \begin{table}[hbt]
        \centering
        \caption{Coeficientes do ajuste polinomial das curvas de calibração dos três sensores.}
        \begin{tabular}{cccc}
        \toprule
        \multirow{2}{*}{\textbf{Coeficiente}} & \multicolumn{3}{c}{\textbf{Valor}} \\ \cline{2-4}
        & \textbf{Sensor X} & \textbf{Sensor Y}\footnote{Coeficientes multiplicados por (-1)} & \textbf{Sensor Z} \\ 
        \midrule
        $K_1$    &  1.9933e-01  &  1.9960e-01  &    1.9985e-01 \\ 
        $K_3$    &  4.1553e-05  &  3.6948e-05  &    3.3025e-05   \\
        $K_5$    & -1.1424e-06  & -1.0317e-06  &   -9.2560e-07 \\
        $K_7$    &  1.0130e-08  &  9.6282e-09  &    8.9590e-09 \\
        $K_9$    & -2.9328e-11  & -2.8003e-11  &   -2.6220e-11  \\
        $K_{11}$ &  3.4512e-14  &  3.3634e-14  &    3.1928e-14\\
        \bottomrule
        \end{tabular}
        \label{tab:sensores}
    \end{table}


\end{document}








